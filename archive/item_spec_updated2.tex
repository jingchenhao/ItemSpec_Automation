\documentclass[11pt, English]{article}
\usepackage{graphicx}	
\usepackage{color, colortbl}
\definecolor{Orange}{rgb}{1.0,0.49,0.0}
\definecolor{brightyellow}{rgb}{1.0,0.67,0.11}
\usepackage {multirow}
\usepackage[utf8]{inputenc}
\usepackage{parskip}
\usepackage[left=0.7in, right=0.7in, top=0.7in, bottom=0.7in]{geometry}
\usepackage{longtable}

\newcommand{\VAR}[1]{}
\newcommand{\BLOCK}[1]{}													
\usepackage[]{color}
\usepackage[table,x11names]{xcolor}
\usepackage[font=bf]{caption}
\usepackage{titlesec}
\usepackage{booktabs}
\usepackage[utf8]{inputenc}
\setlength{\aboverulesep}{0pt}
\setlength{\belowrulesep}{0pt}
\usepackage{enumitem}
\usepackage{fancyhdr}
\usepackage{tocloft,sectsty}


\begin{document}
\thispagestyle{empty}








\textbf{Standard:} MA 6.1.1 Numeric Relationships: Students will demonstrate, represent, and show relationships among fractions, decimals, percents, and integers within the base-ten number system.\\

\begin{longtable}{|p{11.0cm}|p{1.8cm}|p{3.7cm}|}

\hline


\rowcolor{Orange}
        \multicolumn{3}{|p{17.36cm}|}{\textbf{MA 6.1.1.c Compare and order rational numbers both on the number line and not on the number line.
}}\\
        \hline
        \rowcolor{brightyellow}
        \hfil{ALD Level Descriptions}&\hfil{Maximum DOK}&\hfil{Aligned Item Formats}\\
        \hline

\endfirsthead
\multicolumn{3}{@{}l}{\ldots continued}\\\hline



\rowcolor{Orange}
        \multicolumn{3}{|p{17.36cm}|}{\textbf{MA 6.1.1.c Compare and order rational numbers both on the number line and not on the number line.
}}\\ 
        \hline
        \rowcolor{brightyellow}
        \hfil{ALD Level Descriptions}&\hfil{Maximum DOK}&\hfil{Aligned Item Formats}\\
        \hline

\endhead % all the lines above this will be repeated on every page
\hline

\multicolumn{3}{r@{}}{continued \ldots}\\
\endfoot
\hline
\endlastfoot 



\textbf{Level 1 Developing}\newline
• Records comparisons of positive numbers (whole numbers, mixed numbers, fractions, and/or decimals to the tenths, hundredths, or thousandths) using symbols. May include a number line. \newline• Orders three or more positive numbers (whole numbers, mixed numbers, fractions, and/or decimals to the thousandths). May include a number line. \newline• Determines what positive number is between two given positive numbers (whole numbers, mixed numbers, fractions, and/or decimals to the thousandths). May include a number line. \newline
& \hfil{2}
&• Constructed Response\newline• Gap Match or Graphic Gap Match\newline• Graphing\newline• Multiple Choice or Choice Multiple\newline• Text Entry\\
\hline




\textbf{Level 2 On Track}\newline
• Uses symbols to represent comparisons between two rational numbers where at least one value is a decimal to the ten-thousandths or more. \newline• Orders three or more positive numbers (whole numbers, mixed numbers, fractions, and/or decimals) with at least one number being a decimal to the ten-thousandths or a percent. \newline• Uses symbols to represent comparisons of two rational numbers when at least one value is a negative rational number and both values are plotted on a number line. \newline• Orders three rational numbers on a number line when at least one value is a negative rational number. \newline

 & \hfil{2}
 &• Constructed Response\newline• Equation Editor\newline• Gap Match or Graphic Gap Match\newline• Graphing\newline• Hot Text\newline• Multiple Choice or Choice Multiple\newline• Text Entry\\
\hline


 
\textbf{Level 3 College and Career Ready}\newline
• Uses symbols to represent comparisons between two rational numbers where at least one value is a negative rational number. \newline• Orders three or more numbers when at least one value is a negative rational number. \newline

& \hfil{2}
&• Constructed Response\newline• Equation Editor\newline• Gap Match or Graphic Gap Match\newline• Graphing\newline• Hot Text\newline• Multiple Choice or Choice Multiple\newline• Text Entry\\
\hline 


\end{longtable}











\end{document}