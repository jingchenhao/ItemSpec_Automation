\documentclass[11pt, English]{article}
\usepackage{graphicx}	
\usepackage{color, colortbl}
\definecolor{Orange}{rgb}{1.0,0.49,0.0}
\definecolor{brightyellow}{rgb}{1.0,0.67,0.11}
\usepackage {multirow}
\usepackage[utf8]{inputenc}
\usepackage{parskip}
\usepackage[left=0.7in, right=0.7in, top=0.7in, bottom=0.7in]{geometry}
\usepackage{longtable}

\newcommand{\VAR}[1]{}
\newcommand{\BLOCK}[1]{}													
\usepackage[]{color}
\usepackage[table,x11names]{xcolor}
\usepackage[font=bf]{caption}
\usepackage{titlesec}
\usepackage{booktabs}
\usepackage[utf8]{inputenc}
\setlength{\aboverulesep}{0pt}
\setlength{\belowrulesep}{0pt}
\usepackage{enumitem}
\usepackage{fancyhdr}
\usepackage{tocloft,sectsty}


\begin{document}
\thispagestyle{empty}








\textbf{Standard:} MA 6.1.1 Numeric Relationships: Students will demonstrate, represent, and show relationships among fractions, decimals, percents, and integers within the base-ten number system.\\

\begin{longtable}{|p{11.0cm}|p{1.8cm}|p{3.7cm}|}

\hline


\rowcolor{Orange}
        \multicolumn{3}{|p{17.36cm}|}{\textbf{MA 6.1.1.b Represent non-negative whole numbers using exponential notation.
}}\\
        \hline
        \rowcolor{brightyellow}
        \hfil{ALD Level Descriptions}&\hfil{Maximum DOK}&\hfil{Aligned Item Formats}\\
        \hline

\endfirsthead
\multicolumn{3}{@{}l}{\ldots continued}\\\hline



\rowcolor{Orange}
        \multicolumn{3}{|p{17.36cm}|}{\textbf{MA 6.1.1.b Represent non-negative whole numbers using exponential notation.
}}\\ 
        \hline
        \rowcolor{brightyellow}
        \hfil{ALD Level Descriptions}&\hfil{Maximum DOK}&\hfil{Aligned Item Formats}\\
        \hline

\endhead % all the lines above this will be repeated on every page
\hline

\multicolumn{3}{r@{}}{continued \ldots}\\
\endfoot
\hline
\endlastfoot 



\textbf{Level 1 Developing}\newline
• Represents a non-negative whole number less than 100 with a single term in exponential form.\newline• Evaluates a numerical expression with an exponent that represents a non-negative whole number. Must be a number other than a power of 10. \newline• Compares values of non-negative whole numbers when presented in exponential form. Should not require evaluating the expression or rules of exponents but may require rewriting it in an equivalent form.\newline
& \hfil{1}
&• Constructed Response\newline• Equation Editor\newline• Gap Match or Graphic Gap Match\newline• Graphing\newline• Hot Text\newline• Multiple Choice\newline• Text Entry\\
\hline




\textbf{Level 2 On Track}\newline
• Represents a non-negative whole number greater than 100 but not a power of 10 as a single term in exponential form. \newline• Represents a non-negative whole number greater than 100 but not a power of 10 as a single term in exponential form when 1) given the base, determines the exponent, 2) given the exponent, determines the base.\newline

 & \hfil{2}
 &• Constructed Response\newline• Gap Match or Graphic Gap Match\newline• Graphing\newline• Hot Text\newline• Multiple Choice\newline• Text Entry\\
\hline


 
\textbf{Level 3 College and Career Ready}\newline
• Represents more than one way to write exponential form of non-negative whole numbers. Ex: Is there more than one way to write 81 in exponential form? Explain your answer. \newline

& \hfil{1}
&• Constructed Response\newline• Gap Match or Graphic Gap Match\newline• Graphing\newline• Hot Text\newline• Multiple Choice\newline• Text Entry\\
\hline 


\end{longtable}











\end{document}