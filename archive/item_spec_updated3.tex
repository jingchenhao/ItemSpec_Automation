\documentclass[11pt, English]{article}
\usepackage{graphicx}	
\usepackage{color, colortbl}
\definecolor{Orange}{rgb}{1.0,0.49,0.0}
\definecolor{brightyellow}{rgb}{1.0,0.67,0.11}
\usepackage {multirow}
\usepackage[utf8]{inputenc}
\usepackage{parskip}
\usepackage[left=0.7in, right=0.7in, top=0.7in, bottom=0.7in]{geometry}
\usepackage{longtable}

\newcommand{\VAR}[1]{}
\newcommand{\BLOCK}[1]{}													
\usepackage[]{color}
\usepackage[table,x11names]{xcolor}
\usepackage[font=bf]{caption}
\usepackage{titlesec}
\usepackage{booktabs}
\usepackage[utf8]{inputenc}
\setlength{\aboverulesep}{0pt}
\setlength{\belowrulesep}{0pt}
\usepackage{enumitem}
\usepackage{fancyhdr}
\usepackage{tocloft,sectsty}


\begin{document}
\thispagestyle{empty}








\textbf{Standard:} MA 6.1.1 Numeric Relationships: Students will demonstrate, represent, and show relationships among fractions, decimals, percents, and integers within the base-ten number system.\\

\begin{longtable}{|p{11.0cm}|p{1.8cm}|p{3.7cm}|}

\hline


\rowcolor{Orange}
        \multicolumn{3}{|p{17.36cm}|}{\textbf{MA 6.1.1.d Convert among fractions, decimals, and percents using multiple representations.
}}\\
        \hline
        \rowcolor{brightyellow}
        \hfil{ALD Level Descriptions}&\hfil{Maximum DOK}&\hfil{Aligned Item Formats}\\
        \hline

\endfirsthead
\multicolumn{3}{@{}l}{\ldots continued}\\\hline



\rowcolor{Orange}
        \multicolumn{3}{|p{17.36cm}|}{\textbf{MA 6.1.1.d Convert among fractions, decimals, and percents using multiple representations.
}}\\ 
        \hline
        \rowcolor{brightyellow}
        \hfil{ALD Level Descriptions}&\hfil{Maximum DOK}&\hfil{Aligned Item Formats}\\
        \hline

\endhead % all the lines above this will be repeated on every page
\hline

\multicolumn{3}{r@{}}{continued \ldots}\\
\endfoot
\hline
\endlastfoot 



\textbf{Level 1 Developing}\newline
• Converts among fractions and decimals for fractions with denominators of 7 or 9. \newline• Converts between percents and fractions  or percents and decimals for whole percents greater than or equal to 1%. \newline
& \hfil{1}
&• Multiple Choice or Choice Multiple\newline• Text Entry\\
\hline




\textbf{Level 2 On Track}\newline
• Represents quivalent values for given fractions, decimals, and percents for fractions with denominator greater than 10 (other than 100) or percents that include decimals. \newline

 & \hfil{1}
 &• Constructed Response\newline• Gap Match or Graphic Gap Match\newline• Graphing\newline• Hot Text\newline• Multiple Choice or Choice Multiple\newline• Text Entry\\
\hline


 
\textbf{Level 3 College and Career Ready}\newline
• Explains and justifies a conversion between fractions, decimals, and percents, using symbols, visual models, or other representations.\newline

& \hfil{1}
&• Constructed Response\newline• Equation Editor\newline• Gap Match or Graphic Gap Match\newline• Hot Text\newline• Multiple Choice or Choice Multiple\\
\hline 


\end{longtable}











\end{document}